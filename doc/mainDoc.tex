\documentclass{article}

% if you need to pass options to natbib, use, e.g.:
%     \PassOptionsToPackage{numbers, compress}{natbib}
% before loading neurips_2021

% ready for submission
\usepackage[preprint]{neurips_2021}

% to compile a preprint version, e.g., for submission to arXiv, add add the
% [preprint] option:
%     \usepackage[preprint]{neurips_2021}

% to compile a camera-ready version, add the [final] option, e.g.:
%     \usepackage[final]{neurips_2021}

% to avoid loading the natbib package, add option nonatbib:
% \usepackage[nonatbib]{neurips_2021}




\usepackage[utf8]{inputenc} % allow utf-8 input
\usepackage[T1]{fontenc}    % use 8-bit T1 fonts
\usepackage[colorlinks = true, 
		    linkcolor = blue,
		    urlcolor  = blue,
            citecolor = blue,
            anchorcolor = blue]{hyperref}       
% hyperlinks
\usepackage{url}            % simple URL typesetting
\usepackage{booktabs}       % professional-quality tables
\usepackage{amsfonts}       % blackboard math symbols
\usepackage{nicefrac}       % compact symbols for 1/2, etc.
\usepackage{microtype}      % microtypography
\usepackage{xcolor}         % colors
\usepackage{natbib}	

\bibliographystyle{unsrtnat}

\title{Clustering Spotify: How diverse is popular music?}




% The \author macro works with any number of authors. There are two commands
% used to separate the names and addresses of multiple authors: \And and \AND.
%
% Using \And between authors leaves it to LaTeX to determine where to break the
% lines. Using \AND forces a line break at that point. So, if LaTeX puts 3 of 4
% authors names on the first line, and the last on the second line, try using
% \AND instead of \And before the third author name.


\author{%
  Dominik Glandorf\\
  Matrikelnummer 6007407\\
  \texttt{dominik.glandorf@student.uni-tuebingen.de} \\
  \And
  Felix D. Gross\\
  Matrikelnummer 6001480\\
  \texttt{fel.gross@student.uni-tuebingen.de} \\
}

\begin{document}

\maketitle

\begin{abstract}
{Utilizing the Spotify API, we collect the metadata of a large set of tracks, notably the audio features, to assess the cliché of contemporary pop music being all the same.  
  Utilizing pca to reduce the multi-dimensional data, shows already a difference between pop and non pop songs. Further analysis of variance shows, that the variability across certain audio features is indeed less in pop music when compared to non pop music, thereby confirming our initial hypothesis.}
  
\end{abstract}

\section{Introduction}

Music is one of the oldest and most valued forms of communication and is therefore of particular interest. Yet, its underlying structure has been eluded systematic analysis until recently. Comparing numerous features across thousands of songs was to much data to handle with traditional methods of analysis. Accelerated computing and new mathematical methods allow now for systematic inquiry of music. Thus it is only now possible to hold common clichés about music proof to the real world data.\newline One of these clichés is, that pop music is all the same and less diverse then non popular music \citep[see for example][]{serra2012measuring}. Thus, our Hypothesis is, that the variability of pop music is smaller, then the one observed in non popular music.

\section{Method}

\subsection{Data}
Since Spotify is the worlds most used music streaming platform, it promises to be representative for global music listening taste. Thus, the music data was obtained from the Spotify web API \citep{spotify}. For scraping the sample Spotify's search function was used in the following way: For each possible two letter combination the first 500 search results were stored. This led to a data set of \(26^{2} * 500 = 33'800\) songs. After controlling for potential similarity between titles 7'000 songs were removed. After removing duplicates the total sample encompassed \(N_{total} = 28'195\)  songs. 
Spotify provides plenty of features for each song. We concentrated 
on danceability, energy, liveness, loudness, speechiness, acousticness, instrumentalness, tempo, valence and key, as they allow for easy and intelligible  interpretations of the lofty term variability [Further explanation?]. The specific meaning of each of these features can be obtained from the documentation of the Spotify API \citep{spotifyAPIdocu}[table, or still explain here?]. In addition to these features, which provide information about the internal make-up of the song, spotify offered also a "popularity" feature. A value between 1 and 100 was assigned to each song song, with 100 being the most popular. We used this feature to assign our sample to two groups: the pop songs and the non pop songs. As 50 seems to be a suitable size for a hit parade and most of Spotify's own hit play-lists consist out of 50 songs, the \(n_{pop} = 48\) popular songs were assigned to the pop song group and the remainder \(n_{non pop} = 28'147\) to the non pop song group. 

\subsection{Analysis}
To better grasp the multi-dimensional data a PCA was conducted, reducing the ten initial dimensions to only two, which are illustrated in [Figure 1]. The there observed difference was further tested, by comparing the variances of each of the ten features between pop and non pop songs using f-tests. The respective histograms are illustrated in [Figure 2]. 

\subsection{Results}
Tested against the Bonferroni-corrected critical F value [5] of these tests turned out significantly, whereas the rest remained above the threshold. 

\subsection{Discussion}
-other features
-other pop treshold (continous descibtion)

\bibliography{library.bib}

\end{document}
