\documentclass{article}

% if you need to pass options to natbib, use, e.g.:
%     \PassOptionsToPackage{numbers, compress}{natbib}
% before loading neurips_2021

% ready for submission
\usepackage[preprint]{neurips_2021}

% to compile a preprint version, e.g., for submission to arXiv, add add the
% [preprint] option:
%     \usepackage[preprint]{neurips_2021}

% to compile a camera-ready version, add the [final] option, e.g.:
%     \usepackage[final]{neurips_2021}

% to avoid loading the natbib package, add option nonatbib:
%    \usepackage[nonatbib]{neurips_2021}

\usepackage[utf8]{inputenc} % allow utf-8 input
\usepackage[T1]{fontenc}    % use 8-bit T1 fonts
\usepackage[colorlinks=true]{hyperref}       % hyperlinks
\usepackage{url}            % simple URL typesetting
\usepackage{booktabs}       % professional-quality tables
\usepackage{amsfonts}       % blackboard math symbols
\usepackage{nicefrac}       % compact symbols for 1/2, etc.
\usepackage{microtype}      % microtypography
\usepackage{xcolor}         % colors

\title{Clustering Spotify // How bovine is popular music?}

% The \author macro works with any number of authors. There are two commands
% used to separate the names and addresses of multiple authors: \And and \AND.
%
% Using \And between authors leaves it to LaTeX to determine where to break the
% lines. Using \AND forces a line break at that point. So, if LaTeX puts 3 of 4
% authors names on the first line, and the last on the second line, try using
% \AND instead of \And before the third author name.

\author{%
  Dominik Glandorf\\
  Matrikelnummer 6007407\\
  \texttt{dominik.glandorf@student.uni-tuebingen.de} \\
  \And
  Felix Groß\\
  Matrikelnummer 6001480\\
  \texttt{fel.gross@student.uni-tuebingen.de} \\
}

\begin{document}

\maketitle

\begin{abstract}
  \emph{Utilizing the Spotify API, we collect the metadata of a large set of tracks, notably the audio features (e.g. danceability) to assess the cliché of contemporary pop music being all the same.  
  We are aiming to utilize clustering methods (e.g. t-SNE) and create corresponding visualisations. Maybe we also apply data reduction techniques, if they provide further explanatory value. If our curiosity is not quenched by then, we will try to predict the popularity of an artist with the metadata of his ouvre.}
\end{abstract}

You can find a detailed example and instructions on how to use this style file in the attached \texttt{neurip\_2021.tex} file. This includes instructions for how to lay out citations.

\end{document}
